\usepackage[utf8]{inputenc}
\usepackage{amsfonts}
\usepackage{amssymb }
%\usepackage{ngerman}
\usepackage[ngerman,english]{babel}
\usepackage{ upgreek }
%\usepackage{biblatex}
\usepackage{bibgerm}
\usepackage[square,sort,comma,numbers]{natbib}% bibliography style for support for urls in the 
\bibliographystyle{gerplain}
%bibliography
%\usepackage{noweb}
%\pagestyle{plain}% plain without headlines, with something more \pagestyle{noweb}
%\noweboptions{shift,smallcode,longchunks}%german,smallcode,longchunks

\usepackage[official]{eurosym}
\usepackage{graphicx}
\usepackage{caption}
\usepackage{subcaption}
\usepackage{graphviz}
\usepackage[hidelinks]{hyperref}
\usepackage[dvipsnames]{xcolor}

\usepackage{comment}
\usepackage{geometry}
\usepackage{tikz}
\geometry{a4paper, portrait,left=2.5cm, right=2.5cm, top=2cm, bottom=3.5cm}
\usepackage[T1]{fontenc}
\usepackage{lmodern}

\usepackage{array}
\usepackage{listings} % für code 

\usepackage{hyperref}
\hypersetup{
    colorlinks=true,
    linkcolor=black,
    filecolor=magenta,      
    urlcolor=blue,
    citecolor=black,
}

\urlstyle{same}

\usepackage[font=itshape]{quoting}


% Für Headlines und Footer
\usepackage{fancyhdr}

\pagestyle{fancy}
\fancyhf{}
\renewcommand{\footrulewidth}{1pt}
\rhead{Reverse Engineering KVM Switch \includegraphics[width=0.05\textwidth]{logo.png}} % Beschreibung für rechts oben
\lhead{\leftmark} % Beschreibung links oben
\rfoot{Seite \thepage} % Beschreibung rechts unten
\lfoot{Sebastian Feustel \& Matthias Enderlein \\Technische Hochschule Brandenburg } % Beschreibung links unten


\usepackage{glossaries}

\makeglossaries

\newglossaryentry{VM}
{
name=VM,
description={(Virtual Machine) Kapselung eines Rechnersystem innerhalb eines Rechnersystems.}
}

\newglossaryentry{Proxmox VE}
{
name=Proxmox VE,
description={(Proxmox Virtual Environment) Eine Open-Source Virtualisierungsplattform zum Betreiben von virtuellen Maschinen und Containern.}
}

\newglossaryentry{LIGO}
{
name=LIGO,
description={(Laser-Interferometer Gravitationswellen-Observatorium) ist ein Observatorium mit dessen Hilfe erstmal Gravitationswellen nachgewiesen wurden.}
}

\newglossaryentry{AEI}
{
name=AEI,
description={(Albert-Einstein-Institut) es befasst sich mit Grundlagenforschung im Bereich der Gravitationsphysik.}
}

\newglossaryentry{MPDL}
{
name=MPDL,
description={(Max Planck Digital Library) ist die Digitale Bibliothek der Max Planck Gesellschaft}
}

\newglossaryentry{iSCSI}
{
name=iSCSI,
description={(internet Small Computer System Interface) ein Verfahren, zum  Zugriff auf Speicher über das Netzwerk.}
}

\newglossaryentry{LAG}
{
name=LAG,
description={(Link Aggregation) ist die Bündelung mehrerer physischer LAN-Schnittstellen zu einem logischen Kanal.}
}

\newglossaryentry{K8s}
{
name=K8s,
description={(Kubernetes) ist ein System zur Automatisierung der Bereitstellung, Skalierung und Verwaltung von Container.}
}

\newglossaryentry{OpenNebula}
{
name=OpenNebula,
description={ist ein System zum  Managen von Infrastrukturen (Cloud Verwaltung).}
}

\newglossaryentry{vCenter}
{
name=vCenter,
description={(Virtualcenter) ist ein Verwaltungsdienstprogramm für VMware und wird zum Verwalten von virtuellen Maschinen, mehreren ESXi-Hosts verwendet.}
}

\newglossaryentry{ESXi}
{
name=ESXi,
description={(Elastic Sky X) ist ein integrated Bare-Metal-Hypervisor.}
}

\newglossaryentry{SPoF}
{
name=SPoF,
description={(Single Point of Failure) ist ein Bestandteil eines Systems, dessen Ausfall den Ausfall des gesamten Systems nach sich zieht.}
}

\newglossaryentry{VLAN}
{
name=VLAN,
description={(Virtual Local Area Network) ermöglicht die logische Unterteilung von Netzen.}
}

\newglossaryentry{DHCP}
{
name=DHCP,
description={(Dynamic Host Configuration Protocol) ermöglicht die Zuweisung von Netzwerkkonfiguration an Rechner in einem Netzwerk.}
}

\newglossaryentry{CPU}
{
name=CPU,
description={(Central Processing Unit) ist der Hauptprozessor eines Computers.}
}

\newglossaryentry{GB}
{
name=GB,
description={(Gigabyte) ist eine Einheit der Datenspeicherkapazität.}
}

\newglossaryentry{IP-Adresse}
{
name=IP-Adresse,
description={(Internetprotokoll-Adresse) Ist eine Nummer in Computernetzen, sie dient der Zuordenbarkeit.}
}

\newglossaryentry{DNS}
{
name=DNS,
description={(Domain Name System) ist eine Namensauflösung von Domains zu IP-Adressen.}
}

\newglossaryentry{FAI}
{
name=FAI,
description={(Fully Automatic Installation) ein System, das eine automatisierte Installation bereitstellen kann.} 
}

\newglossaryentry{USB}
{
name=USB,
description={(Universal Serial Bus) ist ein Bussystem, dass den Anschluss externer Geräte erlaubt.}
}

\newglossaryentry{ISO-Abbild}
{
name=ISO-Abbild,
description={ein Speicherabbild des Dateisystems einer CD oder DVD.}
}

\newglossaryentry{iDRAC}
{
name=iDRAC,
description={(Integrated Dell Remote Access Controller) erlaubt die Fernwartung von Servern.}
}

\newglossaryentry{VPN}
{
name=VPN,
description={(Virtual Private Network) bietet die Möglichkeit, von außen auf ein bestehendes Netzwerk zuzugreifen.}
}

\newglossaryentry{URL}
{
name=URL,
description={(Uniform Resource Locator) lokalisiert eine Ressource, in einem Netzwerken (häufig das Internet).}
}

\newglossaryentry{LXC}
{
name=LXC,
description={(Linux Containers) ein Verfahren zur Virtualisierung auf Kernelebene.}
}

\newglossaryentry{SD-Karte}
{
name=SD-Karte,
description={(Secure Digital Memory Card, sichere digitale Speicherkarte) ist ein digitales Speichermedium.}
}

\newglossaryentry{IT}
{
name=IT,
description={(Informationstechnik) ist Oberbegriff für alle mit der elektronischen Datenverarbeitung stehende Maßnahmen.}
}

\newglossaryentry{LVM}
{
name=LVM,
description={(Logical Volume Management) bietet eine Abstraktionsebene zwischen Festplatten, Partitionen und Dateisystemen.}
}

\newglossaryentry{ZFS}
{
name=ZFS,
description={(Zettabyte File System) Dateisystem mit vielen Features, wird häufig in Rechenzentren eingesetzt.}
}

\newglossaryentry{PAP}
{
name=PAP,
description={(Programmablaufplan) ist ein Ablaufdiagramm für ein Programm oder Prozess.}
}

\newglossaryentry{LDAP}
{
name=LDAP,
description={(Lightweight Directory Access Protocol) ein Protokoll, dass das Finden von Personen/Ressourcen in einem Netzwerk ermöglicht. Wird häufig zur Authentifizierung genutzt.}
}

\newglossaryentry{Git}
{
name=Git,
description={Ist eine Versionsverwaltung, die häufig für Software verwendet wird.}
}

\newglossaryentry{NFS}
{
name=NFS,
description={ (Network File System) – ist ein  Protokoll, dass den Zugriff auf Dateien über ein Netzwerk erlaubt.}
}